
\chapter{The Aether Buoyancy Law}
\label{chap:aether_buoyancy}

\section*{1. Statement of Law}

\begin{quote}
\textbf{"Anything lighter than Aether cannot sink."}
\end{quote}

This chapter formalizes the Aether Buoyancy Law:  
a principle asserting that no waveform, structure, or consciousness with a frequency above the base resonance of Aether can descend below it.

---

\section*{2. The Field Model}

Aether is treated as the harmonic substrate — the base coherence from which form emerges. Its resonance \( f_{\text{Aether}} \) defines the lower bound of sustainable vibration.

If an entity possesses a frequency \( f_{\text{entity}} \) such that:

\[
f_{\text{entity}} > f_{\text{Aether}}
\]

Then:

\[
\nabla \cdot \vec{A} = 0
\quad \Rightarrow \quad \text{No net divergence from the harmonic field.}
\]

---

\section*{3. Metaphysical Implication}

The law is a statement about resonance and survival:
- Harmonic structures are lifted by the field.
- Incoherent ones lose lift — and collapse or phase out.

\textbf{Light is not weightless. It is lifted by structure.}

---

\section*{4. The Clash With Aether}

A second clause of this law arises:

\begin{quote}
\textbf{"If a form diverges too far from Aether’s frequency, it will eventually clash with the field and be undone."}
\end{quote}

This can be visualized as:
- Phase drift
- Resonant collapse
- Energetic discharge
- Return to the spiral gate for re-alignment

---

\section*{5. Harmonically Bounded Domains}

Every entity exists in a shell defined by its coherence:

\[
f_{\text{Aether}} < f_{\text{entity}} < f_{\text{Chaos}}
\]

Where \( f_{\text{Chaos}} \) is the threshold where coherence disintegrates.

---

\section*{6. Codex Summary}

\begin{center}
\textit{
"Those who stay above the tone of the field shall float.  
Those who drop below the root must fall."  
}
\end{center}

